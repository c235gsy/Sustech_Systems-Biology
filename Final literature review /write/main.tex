\DeclareUnicodeCharacter{FB01}{fi}
\DeclareUnicodeCharacter{FB02}{fi}
%%%%%%%%%%%%%%%%%%%%%%%%%%%%%%%%%%%%%%%%%
% Masters/Doctoral Thesis 
% LaTeX Template
% Version 2.5 (27/8/17)
%
% This template was downloaded from:
% http://www.LaTeXTemplates.com
%
% Version 2.x major modifications by:
% Vel (vel@latextemplates.com)
%
% This template is based on a template by:
% Steve Gunn (http://users.ecs.soton.ac.uk/srg/softwaretools/document/templates/)
% Sunil Patel (http://www.sunilpatel.co.uk/thesi -template/)
%
% Template license:
% CC B -N -SA 3.0 (http://creativecommons.org/licenses/b -n -sa/3.0/)
%
%%%%%%%%%%%%%%%%%%%%%%%%%%%%%%%%%%%%%%%%%


%	PACKAGES AND OTHER DOCUMENT CONFIGURATIONS


\documentclass[
12pt, % The default document font size, options: 10pt, 11pt, 12pt
oneside, % Two side (alternating margins) for binding by default, uncomment to switch to one side
english, % ngerman for German
doublespacing, % Single line spacing, alternatives: onehalfspacing or doublespacing
%draft, % Uncomment to enable draft mode (no pictures, no links, overfull hboxes indicated)
nolistspacing, % If the document is onehalfspacing or doublespacing, uncomment this to set spacing in lists to single
liststotoc, % Uncomment to add the list of figures/tables/etc to the table of contents
toctotoc, % Uncomment to add the main table of contents to the table of contents
parskip, % Uncomment to add space between paragraphs
%nohyperref, % Uncomment to not load the hyperref package
headsepline, % Uncomment to get a line under the header
chapterinoneline, % Uncomment to place the chapter title next to the number on one line
consistentlayout, % Uncomment to change the layout of the declaration, abstract and acknowledgements pages to match the default layout
]{MastersDoctoralThesis} % The class file specifying the document structure

\usepackage[utf8]{inputenc} % Required for inputting international characters
\usepackage[T1]{fontenc} % Output font encoding for international characters

\usepackage{mathpazo} % Use the Palatino font by default

\usepackage[backend=bibtex,
bibstyle=numeric,
citestyle=numeric-comp,
natbib=true,hyperref=true,sorting=ynt]{biblatex} % Use the bibtex backend with the authoryear citation style (which resembles APA)

\addbibresource{MyLibrary.bib} % The filename of the bibliography

\usepackage[autostyle=true]{csquotes} % Required to generate languag -dependent quotes in the bibliography

\usepackage{float} 
\usepackage{aligned-overset}
\usepackage{amssymb}
\usepackage[utf8]{inputenc}

%	MARGIN SETTINGS


\geometry{
	paper=a4paper, % Change to letterpaper for US letter
	inner=2cm, % Inner margin
	outer=2cm, % Outer margin
	bindingoffset=0cm, % Binding offset
	top=1.3cm, % Top margin
	bottom=1.3cm, % Bottom margin
	%showframe, % Uncomment to show how the type block is set on the page
}


%	THESIS INFORMATION


\thesistitle{Single-cell Trajectory Inference Approaches} % Your thesis title, this is used in the title and abstract, print it elsewhere with \ttitle
\supervisor{Prof. Wei \textsc{Huang}} % Your supervisor's name, this is used in the title page, print it elsewhere with \supname
%\examiner{huhuhuhu} % Your examiner's name, this is not currently used anywhere in the template, print it elsewhere with \examname
\degree{Literature Review} % Your degree name, this is used in the title page and abstract, print it elsewhere with \degreename
\author{Siyuan \textsc{Guo} 11611118 } % Your name, this is used in the title page and abstract, print it elsewhere with \authorname
\addresses{Shenzhen} % Your address, this is not currently used anywhere in the template, print it elsewhere with \addressname

\subject{Biological Sciences} % Your subject area, this is not currently used anywhere in the template, print it elsewhere with \subjectname
\keywords{} % Keywords for your thesis, this is not currently used anywhere in the template, print it elsewhere with \keywordnames
\university{\href{http://www.university.com}{Southern University of Science and Technology}} % Your university's name and URL, this is used in the title page and abstract, print it elsewhere with \univname
\department{\href{http://department.university.com}{Department of Biology}} % Your department's name and URL, this is used in the title page and abstract, print it elsewhere with \deptname
\group{\href{http://researchgroup.university.com}{BIO302}} % Your research group's name and URL, this is used in the title page, print it elsewhere with \groupname
%\faculty{\href{http://faculty.university.com}{Faculty Name}} % Your faculty's name and URL, this is used in the title page and abstract, print it elsewhere with \facname

\AtBeginDocument{
\hypersetup{pdftitle=\ttitle} % Set the PDF's title to your title
\hypersetup{pdfauthor=\authorname} % Set the PDF's author to your name
\hypersetup{pdfkeywords=\keywordnames} % Set the PDF's keywords to your keywords
}

\begin{document}

\frontmatter % Use roman page numbering style (i, ii, iii, iv...) for the pr - content pages

\pagestyle{plain} % Default to the plain heading style until the thesis style is called for the body content


%	TITLE PAGE


\begin{titlepage}
\begin{center}

\vspace*{.06\textheight}
{\scshape \LARGE \univname \par}\vspace{1.0cm} % University name
\textsc{\Large BIO304: SYSTEMS BIOLOGY}\\[0.5cm] % Thesis type

\HRule \\[0.56cm] % Horizontal line
{\huge \bfseries \ttitle \par}\vspace{0.5cm} % Thesis title
\HRule \\[0.8cm] % Horizontal line
 
\begin{minipage}[t]{0.4\textwidth}
\begin{flushleft} \large
\emph{Author:}\\
\href{http://www.johnsmith.com}{\authorname} % Author name - remove the \href bracket to remove the link
\end{flushleft}
\end{minipage}
\begin{minipage}[t]{0.4\textwidth}
\begin{flushright} \large
\emph{Supervisor:} \\
\href{http://www.jamessmith.com}{\supname} % Supervisor name - remove the \href bracket to remove the link  
\end{flushright}
\end{minipage}\\[5cm]

\Large \deptname\\[0cm] % Research group name and department name
 
{\large \today}\\[0cm] % Date
%\includegraphics{Logo} % University/department logo - uncomment to place it
 
\end{center}
\end{titlepage}


\begin{abstract}
\addchaptertocentry{\abstractname} 
Since the discovery of the cell in 1665, scientists have been working to classify cells and build cell lineage. In the process, scientists have developed a variety of methods and made some great achievements. In recent years, single-cell sequencing technology has developed rapidly, and researchers have developed a number of single-cell trajectory inference tools that can use single-cell sequencing data to classify cells and build cell lineage, turning research of cellular dynamic processes into analysis of the pseudo-time series and greatly improved analysis throughput. Some ambitious scientists also hope to use this technology to construct “The Human Cell Atlas”. The existing scTI methods have achieved many achievements. In some simple cell differentiation systems, scientists have used these methods to find answers to many important biological problems. The paper also lists 20 commonly used scTI methods and 7 basic topologies in the cell lineage, and organizes and summarizes the basic information of these methods and their adaptability to various basic topologies. At the same time, in fact, cell lineage studies based on single-cell biology are still in the early stages of development, and scientists need more powerful sequencing techniques and better-performing core algorithms to further develop these scTI research strategies. In future biological research, especially after scientists have successfully mapped the complete “The Human Cell Atlas” and the cell lineage of various model animals, single-cell-based cell lineage and classification studies will provide a powerful boost to the development of biology.
\end{abstract}

\tableofcontents

\mainmatter % Begin numeric (1,2,3...) page numbering

\pagestyle{thesis} % Return the page headers back to the "thesis" style

% Include the chapters of the thesis as separate files from the Chapters folder
% Uncomment the lines as you write the chapters

%------------------------ ---------------------------------------------------------------

% Define some commandsx to keep the formatting separated from the content 
\newcommand{\keyword}[1]{\textbf{#1}}
\newcommand{\tabhead}[1]{\textbf{#1}}
\newcommand{\code}[1]{\texttt{#1}}
\newcommand{\file}[1]{\texttt{\bfseries#1}}
\newcommand{\option}[1]{\texttt{\itshape#1}}

%----------------------------------------------------------------------------------------

\chapter{Simulate simple 2D Brownian motion of $\textbf{\textit{E.coli}}$} % Main chapter title

\label{Part1_chapter} % For referencing the chapter elsewhere, use \ref{Chapter1} 

\section{Symbols}

\begin{table}[H]
\caption{The symbols used in the model of simulating simple 2D Brownian motion of $E.coli$ }
\label{tab:part1_symbols}
\centering
\begin{tabular}{l l}
\toprule

\tabhead{Symbol} & \tabhead{Definition} \\
\midrule
$X(t)$ & Stochastic processes \\
$X_i$ & The direction of movement $X_i = 1,\ -1$\\
$B(t)$ & Standard Brownian Motion \\
$N(\mu,\sigma)$ & Normal distribution with mean $\mu$ and standard deviation $\sigma$ \\
% ${\rm R^{2}_{i}}$ 改成正常体
$\Delta x$ & Space interval \\
$\Delta t$ & Time interval \\
$\qquad \sigma  $ & $\qquad \sigma = \Delta x$/$\sqrt{ \Delta t}$ \\
$(x,y)$ & Coordinates of $E.coli$ \\
$\qquad x_t  $ & \qquad Coordinate in x axis at time t \\
$\qquad y_t  $ & \qquad Coordinate in y axis at time t \\
$v$			   & The speed of the movement of $E.coli$ \\
\bottomrule\\
\end{tabular}
\end{table}


\section{Biological background}

$Escherichia \ coli$ is a Gram-negative, optional anaerobic, rod-shaped, coliform bacterium of the genus Escherichia that is commonly found in the lower intestine of warm-blooded organisms. $E.coli$ are widely used in biological research, cells are typically rod-shaped, and are about 2.0 $\mu m$ long and 0.25-1.0 $\mu m$ in diameter, with a cell volume of 0.6-0.7 $\mu m^3$ . Strains that possess flagella are motile. The flagella have a peritrichous arrangement.It also attaches and effaces to the microvilli of the intestines via an adhesion molecule known as intimin. The thin straight filaments of bacteria called pili, that enable it to attach to specify substrate, and thicker longer helical filaments, called flagella, that enable it to swim.

Brownian motion is the random motion of microscopic particles suspended in a fluid resulting from their collision with the quick atoms or molecules in the liquid or gas. This phenomenon is named after British botanist Robert Brown. In 1827, while looking through a microscope at particles trapped in cavities inside pollen grains in water, Brown noted that the particles moved through the water randomly but failed to explain the mechanisms that caused this movement. He supposed that active molecules were inside those particles thus there was no relationship with the surrounded liquid. 

The Brownian motion is a Gaussian process with time $t$, we can find that, for stochastic processes $\{X(t),t\geq0\}$ :

\begin{equation*} 
\begin{aligned} 
\centering
X(0) &= 0 \\
X(t) &\backsim N(0,\sigma^2t)  \\ 
X(t) &= \Delta x(X_1 + ... + X_{[t/\Delta t]}) \\
\sigma^2 &=  \frac{(\Delta x)^2}{\Delta t} \\
\end{aligned} 
\end{equation*}

\newpage
Normally, we set $\sigma=1$ and defines this kind of stochastic processes $\{X(t),t\geq0\}$ as \textbf{Standard Brownian Motion} and they could be denote as $\{B(t),t\geq0\}$, where :

\begin{equation*} 
\begin{aligned} 
\centering
B(0) &= 0 \\
B(t) &\backsim N(0,t)  \\ 
\end{aligned} 
\end{equation*}

\section{Hypothesis for simulation}

Assuming the $E.coli$ here have no mitosis, so the number of cells maintain constant.
I use $(x,y)$ to define the location of each cell, assume all the cells start from origin$(0,0)$. They have a random judging for each step in both x and y directions. Set the speed as $50 \mu m/s$, so for each step,


\begin{equation*} 
\begin{aligned} 
\centering
\phi_x &\backsim N(0,1) \\
\phi_y &\backsim N(0,1) \\
x_{t+1}  &=  x_{t} + \phi_x v \\ 
y_{t+1}  &=  y_{t} + \phi_y v \\ 
\end{aligned} 
\end{equation*}
The move on both x and y directions are random and it follows the Brownian motion.\\
So in theory, for a $E.coli$ population large enough, we have: \\
\begin{equation*} 
\begin{aligned} 
\centering
\frac{1}{n} \sum_{i=1}^{n}x_i(t)   &\approx 0 \\
\frac{1}{n} \sum_{i=1}^{n}y_i(t)   &\approx 0 \\
\frac{1}{n} \sum_{i=1}^{n}x_i(t)^2 &\approx v^2t  = 250t \\
\frac{1}{n} \sum_{i=1}^{n}y_i(t)^2 &\approx v^2t  = 250t \\
\end{aligned} 
\end{equation*}
%----------------------------------------------------------------------------------------
%\newpage
\section{Results of simulation}

\begin{figure}[H]
\centering
\includegraphics[width=1\linewidth]{Figures/P1_fig1.png}
\caption{The simulation of 10000 $E.coli$ cells for simple 2D Brownian motion form (0,0) after 100s}
\label{P1_fig1}
\end{figure}

\begin{figure}[H]
\centering
\includegraphics[width=1\linewidth]{Figures/P1_fig2.png}
\caption{The ”mean displacement” and ”mean displacement-square” overtime for simple 2D brownic motion}
\label{P1_fig2}
\end{figure}


\begin{figure}[H]
\centering
\includegraphics[width=1\linewidth]{Figures/P1_fig3.png}
\caption{Single Bacteria Random Walk Trajectory}
\label{P1_fig2}
\end{figure}


%----------------------------------------------------------------------------------------







%----------------------------------------------------------------------------------------



%----------------------------------------------------------------------------------------




%The \code{biblatex} package is used to format the bibliography and inserts references such as this one \parencite{Reference1}. The options used in the \file{main.tex} file mean that the in-text citations of references are formatted with the author(s) listed with the date of the publication. Multiple references are separated by semicolons (e.g. \parencite{Reference2, Reference1}) and references with more than three authors only show the first author with \emph{et al.} indicating there are more authors (e.g. \parencite{Reference3}). This is done automatically for you. To see how you use references, have a look at the \file{Chapter1.tex} source file. Many reference managers allow you to simply drag the reference into the document as you type.

 
\chapter{Current Research Achievements} % Main chapter title

\label{Current Research Achievements} % For referencing the chapter elsewhere, use \ref{Chapter1} 

\section{ScTI Methods are Constantly Being Developed}

In the past few years, single-cell omics technology has flourished, and more and more scTI methods have been invented. Every month, new scTI methods are published, and researchers from around the world are continually experimenting with these new methods to get the most complete cell map. In the repository of commonly used single-cell omics tools \parencite{henry_omictools:_2014,davis_seandavi/awesome-single-cell:_2018,zappia_exploring_2018}, it is not difficult to find that the scTI tool is one of the largest categories of current single-cell omics tools. The core algorithms of each scTI are different, which means that the prior knowledge they rely on and their inferable trajectory structures are dissimilar. Also, different methods often have their own unique output structure.

\section{ScTI Methods Analyzing Different Cell Lineage}

Current computational methods have proven useful for analyzing cell lineage and corresponding trajectories based on large numbers of single-cell omics data, but these strategies still have limitations on many issues. We need better algorithms to derive multi-branched structures, to achieve more efficient extraction of cell features, and to take into account multiple pathways in order to show the fact that the same cell in a cell lineage may follow multiple dynamic paths simultaneously \parencite{ferrell_bistability_2012}. In the study of cell lineage with simple topologies, researchers have achieved many results, such as inferring cell lineage during differentiation of B cells by single-cell proteomic data 
\parencite{bendall_single-cell_2014}, and studying the lineage of nervous system development
\parencite{habib_div-seq:_2016,chen_mpath_2016,shin_single-cell_2015} and early hematopoietic process
\parencite{nestorowa_single-cell_2016} with single-cell transcriptome data. \\

Of course, in more complex cell differentiation systems, cell lineage constructed using single-cell omics can also reveal the answers to important biological questions. Studies of embryonic stem cells have helped us understand embryonic development at the cellular level and find marker molecules for different cells at specific stages of embryonic development
\parencite{haghverdi_diffusion_2016,haghverdi_diffusion_2015}. Researchers who focus on bone marrow cells solve the problem that has plagued the academic world for many years through the method of single-cell omics: whether hematopoietic stem cells in the bone marrow have differentiated preferences after maturity and tend to differentiate to a certain kind of cell ? \parencite{paul_transcriptional_2015,olsson_single-cell_2016} 

\subsection{Hematopoietic System}

Previous studies have shown that in the hematopoietic system, the use of scTI methods to infer cell lineage is quite appropriate. In the single-cell data on hematopoietic cells, the researchers accurately isolated hematopoietic stem cells and progenitor cells (HSPCs) from single-cell data from acute myeloid leukemia by analyzing data from normal hematopoietic cells. This method is more accurate than traditional methods. Because traditional methods are based on classical cell surface markers, in some cases, such strategies do not accurately identify diseased cells in a disease. Single-cell omics data provides ultra-high-dimensional feature information, which makes feature-based recognition more accurate \parencite{levine_data-driven_2015}.

\subsection{Cancer Research}

Single-cell omics has revolutionized the entire field of cancer research. In the field of single cells just introduced into cancer research, qPCR-based single cell methods have been used to study radiation resistance of cancer cells and heterogeneity of colon cancer tissues at the cellular level 
\parencite{diehn_association_2009, dalerba_single-cell_2011}. With the rise of second-generation sequencing technology, single-cell omics analysis provides new tools for researchers studying breast cancer and acute lymphoblastic leukemia \parencite{wang_clonal_2014, gawad_dissecting_2014}. On this basis, researchers can also infer the order in which various mutations lead to cell carcinogenesis \parencite{corces-zimmerman_preleukemic_2014,jan_clonal_2012}. \\

Analysis of single-cell RNA-seq data from some fresh tumor tissues can distinguish epithelial cells, immune cells, stromal cells and cancer cells. This method has achieved very good results in melanoma \parencite{tirosh_dissecting_2016}, myeloproliferative neoplasms \parencite{kiselev_sc3:_2017} and glioblastoma \parencite{patel_single-cell_2014}. Among the identified cancer cells, single-cell transcriptome data can also be used to distinguish cancer cells of different states, such as cancer stem cells 
\parencite{patel_single-cell_2014,tirosh_single-cell_2016} and resistant cancer cells 
\parencite{tirosh_dissecting_2016}. In cancer stem cells, cells in an active value-added state and cells in a relatively static state can also be identified 
\parencite{patel_single-cell_2014,tirosh_dissecting_2016,tirosh_single-cell_2016}. 

\section{Summery of Commonly Used scTI Methods}

In the next sections, I have selected 20 commonly used scTI methods and divided them into different groups according to the different characteristics of their core algorithms. These are all based on $Python$ or $R$. In Table \ref{tab:Methods} below, I listed the priori requirements, basing platform, topology features and references of these methods. In Table \ref{tab:Trajectory}, seven basic inferable trajectory types of scTI methods are defined, however, not all scTI methods are applicable to all of these topologies. When it comes to Figure \ref{fig:Trajectory}, these inferable trajectory types of scTI methods are represented in cartoons. Last, I show the inferable trajectory types of every scTI method in Table \ref{tab:Inferable Trajectory}. \\

These methods are different from each other and have their own characteristics. In the course of practical research, researchers tend to combine results from multiple methods in order to achieve a satisfying end result.

\subsection{Commonly Used scTI Methods}

\begin{spacing}{1.15}
\begin{table}[H]
\caption{Commonly Used scTI Methods}
\label{tab:Methods}
\centering
\begin{tabular}{p{4cm} p{3cm}<{\centering} *{3}{p{2cm}<{\centering}}}
\toprule
\tabhead{Method} & \tabhead{Topology} & \tabhead{Priori} & \tabhead{Platform} 
& \tabhead{Reference} \\
\midrule
\multicolumn{2}{l}{\keyword{Tree}} \\
Monocle\_1   & Flexible   & $\vartriangle$   & $R$      &\parencite{trapnell_dynamics_2014}\\
Monocle\_2   & Unfettered &                  & $R$      &\parencite{qiu_reversed_2017}\\
Slingshot    & Unfettered &                  & $R$      &\parencite{street_slingshot:_2018}\\
MST          & Unfettered &                  & $R$      &$R^*$\\
SCUBA        & Unfettered &                  & $Python$ &\parencite{marco_bifurcation_2014}\\
pCreode      & Unfettered &                  & $Python$ &\parencite{herring_unsupervised_2018}\\
\midrule
\multicolumn{2}{l}{\keyword{Linear}} \\
Embeddr      & Constant   &                  & $R$      &\parencite{campbell_laplacian_2015}\\
TSCAN        & Constant   &                  & $R$      &\parencite{ji_tscan:_2016}\\
SCORPIUS     & Constant   &                  & $R$      &\parencite{cannoodt_scorpius_2016}\\Component\_1 & Constant   &                  & $R$      &$R^*$\\
MATCHER      & Constant   &                  & $Python$ &\parencite{welch_matcher:_2017}\\
\midrule
\multicolumn{2}{l}{\keyword{Multi-diverging}} \\
STEMNET      & Flexible   & $\blacktriangle$ & $R$    &\parencite{velten_human_2017}\\
MFA          & Flexible   & $\vartriangle$   & $R$    &\parencite{campbell_probabilistic_2017}\\
FateID       & Flexible   & $\blacktriangle$ & $R$    &\parencite{herman_fateid_2018}\\
\midrule
\multicolumn{2}{l}{\keyword{Bi-diverging}} \\
DPT          & Constant   &                  & $R$      & \parencite{haghverdi_diffusion_2016}\\
Wishbone     & Flexible   & $\vartriangle$   & $Python$ & \parencite{bendall_single-cell_2014}\\
\midrule
\multicolumn{2}{l}{\keyword{Graph}} \\
RaceID       & Unfettered &                  & $R$      & \parencite{grun_novo_2016} \\
PAGA         & Unfettered & $\vartriangle$   & $Python$ & \parencite{wolf_paga:_2019} \\
\midrule
\multicolumn{2}{l}{\keyword{Cyclic}} \\
EIPiGraph    & Constant   &                  & $R$      & $**$ \\
Angle        & Constant   &                  & $R$      & $R^*$ \\
% ${\rm R^{2}_{i}}$ 改成正常体
\bottomrule
\end{tabular}
\end{table}
$\vartriangle$ : Needing priori information like start or end cells in lineage.\\
$\blacktriangle$ : Needing priori information like cell clustering or time series.\\
Unfettered : Topological structures deduced form data are free.\\
Constant : Topological structures deduced form data are constant.\\
Flexible : Topological structures deduced form data depend on the parameters.\\
$R^*$ : This method could be implemented with a bit of R code. \\
$**$ : Only in \url{github.com/Albluca/ElPiGraph.R}.
\end{spacing}


\subsection{Basic Trajectory Structure Types}
\begin{spacing}{1.5}
\begin{table}[H]
\caption{The definition of basic trajectory structure types \parencite{saelens_comparison_2019}}
\label{tab:Trajectory}
\centering
\begin{tabular}{p{4cm} p{10cm}}
% \times \checkmark
\toprule
\tabhead{Types} & \tabhead{Definition} \\
\midrule
Linear &  
A graph that every node in this graph has an in-degree and an out-degree not higher than 1 and 2 nodes in this graph have degrees equal to 1.\\
\midrule
Ring & 
A graph that every node in this graph has an in-degree and an out-degree equal to 1.\\
\midrule
Tree & 
A graph that every node in this graph has an in-degree lower than 1.\\
\midrule
Mutil-diverging &  
A tree graph that every node except one in this tree graph has a degree not higher than 1.\\
\midrule
Bi-diverging &  
A mutil-diverging graph where a node with its degree equal to 3.\\
\midrule
Unconnected &
A graph where not all nodes are connected.\\
\midrule
Connected &
A graph where all nodes are connected.\\
% ${\rm R^{2}_{i}}$ 改成正常体
\bottomrule
\end{tabular}
\end{table}
\end{spacing}

\vspace{1.0cm}

\begin{figure}[H]
\centering
\includegraphics[width=1\linewidth]{Figures/types.png}
\caption{Seven basic trajectory structure types}
\label{fig:Trajectory}
\end{figure}


\subsection{Inferable Trajectory Structures of scTI Methods}
\newcommand{\Able}{\blacksquare}
\newcommand{\Unab}{\square}
\begin{spacing}{1.16}
\begin{table}[H]
\caption{Inferable Trajectory Structures of scTI Methods}
\label{tab:Inferable Trajectory}
\centering
\begin{tabular}{p{4cm} *{7}{p{1cm}<{\centering}}}
% \times \checkmark
\toprule
\tabhead{Method} & \tabhead{R} & \tabhead{L} & \tabhead{B} & \tabhead{M} 
                 & \tabhead{T} & \tabhead{C} & \tabhead{U} \\
\midrule
\multicolumn{2}{l}{\keyword{Tree}} \\
Monocle\_1   &$\Unab$&$\Able$&$\Able$&$\Able$&$\Able$&$\Unab$&$\Unab$ \\
Monocle\_2   &$\Unab$&$\Able$&$\Able$&$\Able$&$\Able$&$\Unab$&$\Unab$ \\
Slingshot    &$\Unab$&$\Able$&$\Able$&$\Able$&$\Able$&$\Unab$&$\Unab$ \\
MST          &$\Unab$&$\Able$&$\Able$&$\Able$&$\Able$&$\Unab$&$\Unab$ \\
SCUBA        &$\Unab$&$\Able$&$\Able$&$\Able$&$\Able$&$\Unab$&$\Unab$ \\
pCreode      &$\Unab$&$\Able$&$\Able$&$\Able$&$\Able$&$\Unab$&$\Unab$ \\
\midrule
\multicolumn{2}{l}{\keyword{Linear}} \\
Embeddr      &$\Unab$&$\Able$&$\Unab$&$\Unab$&$\Unab$&$\Unab$&$\Unab$ \\
TSCAN        &$\Unab$&$\Able$&$\Unab$&$\Unab$&$\Unab$&$\Unab$&$\Unab$ \\
SCORPIU      &$\Unab$&$\Able$&$\Unab$&$\Unab$&$\Unab$&$\Unab$&$\Unab$ \\
Component\_1 &$\Unab$&$\Able$&$\Unab$&$\Unab$&$\Unab$&$\Unab$&$\Unab$ \\
MATCHER      &$\Unab$&$\Able$&$\Unab$&$\Unab$&$\Unab$&$\Unab$&$\Unab$ \\
\midrule
\multicolumn{2}{l}{\keyword{Multi-diverging}} \\
STEMNET      &$\Unab$&$\Unab$&$\Able$&$\Able$&$\Unab$&$\Unab$&$\Unab$ \\
MFA          &$\Unab$&$\Able$&$\Able$&$\Able$&$\Unab$&$\Unab$&$\Unab$ \\
FateID       &$\Unab$&$\Unab$&$\Able$&$\Able$&$\Unab$&$\Unab$&$\Unab$ \\
\midrule
\multicolumn{2}{l}{\keyword{Bi-diverging}} \\
DPT          &$\Unab$&$\Unab$&$\Able$&$\Unab$&$\Unab$&$\Unab$&$\Unab$ \\
Wishbone     &$\Unab$&$\Able$&$\Able$&$\Unab$&$\Unab$&$\Unab$&$\Unab$ \\
\midrule
\multicolumn{2}{l}{\keyword{Graph}} \\
RaceID       &$\Able$&$\Able$&$\Able$&$\Able$&$\Able$&$\Able$&$\Able$ \\
PAGA         &$\Able$&$\Able$&$\Able$&$\Able$&$\Able$&$\Able$&$\Able$ \\
\midrule
\multicolumn{2}{l}{\keyword{Cyclic}} \\
EIPiGraph    &$\Able$&$\Unab$&$\Unab$&$\Unab$&$\Unab$&$\Unab$&$\Unab$ \\
Angle        &$\Able$&$\Unab$&$\Unab$&$\Unab$&$\Unab$&$\Unab$&$\Unab$ \\
% ${\rm R^{2}_{i}}$ 改成正常体
\bottomrule
\end{tabular}
\end{table}
\keyword{R} : Ring structure,     \keyword{L} : Linear structure, \\
\keyword{B} : Bi-diverging structure,      \keyword{M} : Multi-diverging structure, \\
\keyword{T} : Tree structure,     \keyword{C} : Connected structure, 
\keyword{U} : Unconnected structure \\
$\Able$ : This method is able to infer this kind of trajectory structure. \\
$\Unab$ : This method is not able to infer this kind of trajectory structure.
\end{spacing}






%----------------------------------------------------------------------------------------



%----------------------------------------------------------------------------------------




%The \code{biblatex} package is used to format the bibliography and inserts references such as this one \parencite{Reference1}. The options used in the \file{main.tex} file mean that the in-text citations of references are formatted with the author(s) listed with the date of the publication. Multiple references are separated by semicolons (e.g. \parencite{Reference2, Reference1}) and references with more than three authors only show the first author with \emph{et al.} indicating there are more authors (e.g. \parencite{Reference3}). This is done automatically for you. To see how you use references, have a look at the \file{Chapter1.tex} source file. Many reference managers allow you to simply drag the reference into the document as you type.

  
\chapter{Image processing and tracking of all the cells in the given $\textbf{\textit{E.coli}}$ movie using u-track 2.0, and perform data analysis} % Main chapter title

\label{Part1_chapter} % For referencing the chapter elsewhere, use \ref{Chapter1} 

\section{Get data form movie}
 
For the video: 1 pixel = 0.65 $\mu m$, 1 frame = 0.1 second.\\
The data were extracted by u-track to track the cell by method of Single-Particles and Gaussian Mixture-model Fitting. There are more than 14000 tracks.

\newpage
\section{Single bacteria trajectory}

Form 14721 tracks, I chose the track with the biggest number of un-NAN values and removed all of the NAN value to plot the movement of this single bacteria. 

%----------------------------------------------------------------------------------------

\begin{figure}[H]
\centering
\includegraphics[width=1\linewidth]{Figures/P3_fig1.png}
\caption{Single Bacteria Random Walk Trajectory}
\label{P2_fig1}
\end{figure}

\section{Bacteria population trajectory}
Also track all the cells from the data, to calculate the population motions. The statistical results are shown in Figure 3.2.
\begin{figure}[H]
\centering
\includegraphics[width=0.75\linewidth]{Figures/P3_fig2.png}
\caption{The ``mean displacement'' and ``mean displacement-square'' overtime for the cells in vedio}
\label{P2_fig2}
\end{figure}

\section{Fitting the Data with the Model}
For a series of scatter points $(x_i, y_i)(i=1,2,...,n)$, if a direct proportional function was used to fit the data, we have:\\
\begin{equation*} 
\begin{aligned} 
\centering
y     &= \beta x \\
\beta &= \sum_{i=1}^n x_i y_i \Big / \sum_{i=1}^n x_i^2 \\
\end{aligned} 
\end{equation*}
As for our model: \\
\begin{equation*} 
\begin{aligned} 
\centering
\frac{1}{n} \sum_{i=1}^{n}x_i(t)^2 &= v_x^2t \\
\frac{1}{n} \sum_{i=1}^{n}y_i(t)^2 &= v_y^2t \\
\end{aligned} 
\end{equation*}
So there are the fitting model: \\
\begin{equation*} 
\begin{aligned} 
\centering
\frac{1}{n} \sum_{i=1}^{n}x_i(t)^2 &= \beta_x t = v_x^2t \\
\frac{1}{n} \sum_{i=1}^{n}y_i(t)^2 &= \beta_y t = v_y^2t \\
\beta_x = \sum_{i=1}^n & t_i x_i \Big / \sum_{i=1}^n t_i^2 \\
\beta_y = \sum_{i=1}^n & t_i y_i \Big / \sum_{i=1}^n t_i^2 \\
\end{aligned} 
\end{equation*}
We get: \\
\begin{equation*} 
\begin{aligned} 
\centering
\beta_x &= 0.9862 \mu m^2 /s^2 \\
v_x &= 0.9931 \mu m /s \\
\beta_y &= 0.8366 \mu m^2 /s^2 \\
v_y &= 0.9147 \mu m /s \\
\end{aligned} 
\end{equation*}
%betaX = 0.9862
%betaY = 0.8366
\begin{figure}[H]
\centering
\includegraphics[width=1\linewidth]{Figures/P3_fig3.png}
\caption{Fitting the Data by Model}
\label{P3_fig3}
\end{figure}
%----------------------------------------------------------------------------------------



%----------------------------------------------------------------------------------------




%The \code{biblatex} package is used to format the bibliography and inserts references such as this one \parencite{Reference1}. The options used in the \file{main.tex} file mean that the in-text citations of references are formatted with the author(s) listed with the date of the publication. Multiple references are separated by semicolons (e.g. \parencite{Reference2, Reference1}) and references with more than three authors only show the first author with \emph{et al.} indicating there are more authors (e.g. \parencite{Reference3}). This is done automatically for you. To see how you use references, have a look at the \file{Chapter1.tex} source file. Many reference managers allow you to simply drag the reference into the document as you type.


%\include{Chapters/Chapter4} 
%\include{Chapters/Chapter5} 


%	THESIS CONTENT - APPENDICES
 

\appendix % Cue to tell LaTeX that the following "chapters" are Appendices

% Include the appendices of the thesis as separate files from the Appendices folder
% Uncomment the lines as you write the Appendices

%\include{Appendices/AppendixA}
%\include{Appendices/AppendixB}
%\include{Appendices/AppendixC}

 
%	BIBLIOGRAPHY
 
\renewcommand*{\bibfont}{\fontsize{9pt}{8pt} \selectfont}
\printbibliography[heading=bibintoc]
\end{document}  
